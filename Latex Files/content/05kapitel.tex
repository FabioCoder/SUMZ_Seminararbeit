%!TEX root = ../dokumentation.tex

\chapter{OpenUI5}
\vskip -3em
\textbf{Autor: Sebastian Greulich}
\label{ch:openUI5}

\section{Allgemeines}

\subsection{Einführung in das Framework}\label{sec:oEinf}

Das dritte, im Rahmen dieser Seminararbeit betrachtete clientseitige Webframework heißt OpenUI5. Es ist ein Open-Source-Projekt der SAP SE und wurde im Dezember 2013 unter der Apache-Lizenz 2.0 veröffentlicht. Diese Lizenz erlaubt das freie Verwenden, Verteilen und Modifizieren der Software. Durch diese Eigenschaften ist sie mit der GNU General Public License 3 kompatibel. Das Pendant zu OpenUI5 ist SAPUI5, dies ist die proprietäre Version von OpenUI5. SAPUI5 hat den gleichen Funktionsumfang wie OpenUI5 und wird von der SAP SE kostenfrei an ihre Kunden vertrieben. Der Vorteil von SAPUI5 ist ein umfangreicher Softwaresupport welcher für OpenUI5 nicht verfügbar ist. Da man für den Einsatz von SAPUI5 SAP-Kunde sein muss wird dieses Produkt im Rahmen dieser Seminararbeit nicht weiter betrachtet.\autocites[vgl.][8]{Magnucki2017}

Das OpenUI5-Framework basiert auf HTML5, CSS3 und JavaScript (TypeScript ist nicht vorgesehen, da es in keiner offiziellen Dokumentation erwähnt wird). Die OpenUI5-Standardsteuerelemente sind laut Dokumentation unter anderem mit folgenden Browserversionen kompatibel: Internet Explorer ab Version 9, Firefox ab Version 17 und Google Chrome ab Version 1. Beim Scripting hängt die Kompatibilität zum Browser an der jeweils verwendeten ECMAScript-Version. Es wird kein offizielles Tool zum Transpilieren bereitgestellt. \autocites[vgl.][6]{SAP2013} Daraus folgt, dass der Entwickler bei der ECMAScript-Version darauf achten sollte, welche Browserzielgruppe er ansprechen will.

\subsection{Auswahl der Entwicklungsumgebung}\label{sec:oEntw}

Prinzipiell kann für die Entwicklung von OpenUI5-Anwendungen jeder beliebige Texteditor verwendet werden. Für größere Webentwicklungsprojekte wird jedoch auf der OpenUI5-Dokumentationshomepage die Entwicklungsumgebung Eclipse empfohlen. Für diese wurde von der SAP das „SAPUI5 Application Development“-AddOn entwickelt. In diesem AddOn sind alle notwendigen Bibliotheken zur OpenUI5-Entwicklung enthalten. Darüber hinaus dient es als Assistent zur Erstellung neuer UI5-Entwicklungsprojekte. Der Assistent erstellt nach seinem Durchlauf eine angepasste Dateistruktur auf der man als Entwickler aufbauen kann.\autocites[vgl.][109\psqq]{Antolovic2014}


\subsection{Typische Einsatzgebiete}

Für OpenUI5 existiert eine sehr umfassende und übersichtliche Dokumentation unter \url{www.openui5.hana.ondemand.com/#/api}. Diese umfasst viele Anwendungsbeispiele aus der Praxis. Hierbei ist es möglich, den verwendeten Code zu analysieren und in der eigenen Anwendung wiederzuverwenden. OpenUI5 wurde von der SAP entwickelt um die veralteten SAP ERP-Transaktionen zu ersetzen. In diesem Umfeld hat OpenUI5 auch die größten Stärken. Beispielsweise ist es sehr komfortabel möglich Single-Screen-Geschäftsanwendungen zu erstellen. Diese Anwendungen haben ihren Fokus zumeist auf dem Anzeigen, Pflegen und Auswerten von Daten aus einer Datenbank. Jedoch ist es aufgrund der Verwendung von HTML5 auch möglich, verschiedenste multimediale Inhalte wie beispielsweise Audio oder Video wiederzugeben. Einen weiteren Vorteil von OpenUI5 stellt die Responiveness der Oberfläche dar. Das Framework beinhaltet Funktionen, welche dafür sorgen, dass die Weboberfläche auf jeder Geräteklasse optimal dargestellt wird. Bei der Datenanbindung beherrscht OpenUI5 unter anderem die OData-Technologie, diese ermöglicht es Daten im JSON-Format asynchron über AJAX zu übertragen. Das auf HTTP basierende OData-Protokoll ist ein von Microsoft entwickelter offener Standard zur Datenübertragung. Der letzte Vorteil von OpenUI5 ist die mögliche Verwendung des MVC-Modells. Dieses vereinfacht die Übersichtlichkeit und Wartbarkeit der Webanwendung.\autocites[vgl.][26\psqq]{Magnucki2017}


\section{Architektur}

\subsection{Einbinden von OpenUI5}\label{sec_oGrund}

In diesem Abschnitt wird beschrieben wie man OpenUI5 ohne die vom Assistenten generierten MVC-Dateifragmente initialisiert (siehe \autoref{lst:UI5Einbinden}). Allgemein ist OpenUI5, wie in den technischen Grundlagen in \autoref{sec:FWTypen} beschrieben, ein clientseitiges JavaScript White-Box-Framework. Dies bedeutet, man kann das Framework mittels \texttt{<script>}-Befehl in ein HTML-Dokument importieren und anpassen. Hierbei können wichtige Interfaceparameter übergeben werden, wie beispielsweise das zu verwendende Theme. Dies kann man in \autoref{lst:UI5Einbinden} anschaulich sehen.

\begin{lstlisting}[caption=Beispiel für das Einbinden von OpenUI5, label=lst:UI5Einbinden, language=HTML]
<!DOCTYPE html>
	<html>
	...
		<head>
			<script id="sap-ui-bootstrap"
				src=https://openui5.hana.ondemand.com/resources/sap-ui-core.js
				data-sap-ui-theme="sap_belize">
			</script>
		</head>
	...
	</html>
\end{lstlisting}

Nachdem die Bibliothek in das Dokument integriert wurde kann man in einem JavaScript-Teil des HTML-Dokumentes verschiedene UI-Elemente generieren und diese auf der Webseite anzeigen lassen. Dieses Verfahren eignet sich jedoch aufgrund der schlechten Strukturierungsmöglichkeiten nur für sehr kleine und kompakte Webanwendungen mit geringem Funktionsumfang.\autocites[vgl.][136\psqq]{Antolovic2014}

\subsection{Dateifragmente}\label{sec:oDateien}

Wie im vorherigen \autoref{sec_oGrund} erwähnt, existiert für OpenUI5 ein AddOn, welches beim Anlegen des Projektes alle wichtigen Bibliotheken und eine MVC-Dateistruktur erstellt. Die wichtigsten Dateielemente dieser Struktur lauten:

\begin{itemize}
	\item Index.html
	\item Component.js
	\item View.xml
	\item Controller.js
	\item I18n.properties
\end{itemize}

Auf die Funktion und Struktur der Elemente wird im Verlauf dieses Kapitels detailliert eingegangen. Hierbei ist es wichtig, dass die beschriebenen JavaScript- und XML-Dateien aus mehreren Komponenten bestehen und auch anders benannt werden können. Die folgenden Dateien werden lediglich als Beispiele für eine empfohlene Struktur herangezogen. 

Die index.html-Datei ist der erste Schritt für den Browser beim Öffnen einer Webseite. Sie ist zwingend nötig und darf nicht umbenannt oder gelöscht werden. In der Datei index.html werden allgemeine Parameter wie beispielsweise Dateipfade oder Kompatibilitätsinformationen festgelegt. Darüber hinaus wird von ihr auf die anderen Bestandteile der Webanwendung verwiesen.

In der Component.js werden alle Datenverbindungen beschrieben. Hierzu gehören beispielsweise einfach eingebundene JSON-Dateien oder auch Verbindungen zu einem ODATA-Webservice welcher dem UI5-Frontend Daten zur Anzeige bereitstellt. Hierbei kann OpenUI5 im Standard sowohl JSON-, als auch XML-Dateien verarbeiten. Weiter wird in der Component.js festgelegt welche Art der Datenbindung zum Webservice besteht. Möglich ist hierbei das bloße Anzeigen (unidirektionale Bindung), zusätzlich aber auch das Verändern auf der Datenbank (bidirektionale Bindung).

Die View.xml ist von der Struktur eine XML-Datei. Wie ihr Name schon aussagt wird in ihr die visuelle Anzeige beschrieben. Die XML-Datei eignet sich in ihrem Aufbau sehr gut um eine grafische Oberfläche mit vorgegebenen Steuerungselementen zu strukturieren. In ihr kann man nämlich verschiedene Anzeigeelemente klar erkennen und voneinander trennen. Dazu gehören zum Beispiel Header, Footer oder einzelne Container mit verschiedenem Inhalt. Mittels dem im OpenUI5-Standard enthaltenen Router ist es möglich, verschiedene Views miteinander zu verbinden und somit eine Webanwendung mit mehreren Unterseiten zu schaffen. Die View in UI5 ermöglicht darüber hinaus auch die einfache Datenbindung zur Datenbank ohne zusätzliche Programmierung im Controller. Mit dem folgenden \autoref{lst:UI5View}, einer Produktliste, wird die große Stärke von OpenUI5 nochmals unterstrichen. Mit einem einzigen Befehl \texttt{items="{/ProductCollection}"} kann die Datenquelle angebunden werden.


\begin{lstlisting}[caption=Beispiel OpenUI5 View, label=lst:UI5View, language=XML]
<List
id="ProductList"
items="{/ProductCollection}"
includeItemInSelection="true">
	<StandardListItem
	title="{Name}"
	description="{ProductId}"
	icon="{ProductPicUrl}"/>
</List>
\end{lstlisting}

Die Datei Controller.js enthält die programmspezifische Logik der UI5-Webanwendung. In ihr können mittels JavaScript die gewünschten Funktionen implementiert werden. Eine typische Funktion stellt beispielsweise das Öffnen eines Dialogfensters beim Drücken eines zuvor in der View definierten Buttons dar. Somit fügt der Controller der View Funktionen hinzu, welche über einfache Tabellenanzeigen oder -änderungen hinausgehen. Im \autoref{lst:UI5Controller} wird ein Szenario veranschaulicht bei welchem auf Knopfdruck ein MessageToast mit Blindtext erscheint. Es ist zudem auch möglich, mehrere Controller zu verwenden und in diese weitere Frameworks zur Benutzung zu laden. Als mögliches Anwendungsbeispiel wäre hierbei das Einbinden eines Barcodescannerframeworks zu nennen(zum Beispiel quagga.js). Dadurch wird im Browser ermöglicht, mit einer Handykamera industrielle Barcodes lesen zu können. 

\begin{lstlisting}[caption=Beispiel OpenUI5 Controller, label=lst:UI5Controller, language=Javascript]
handleMessageToastPress: function(oEvent) {
	var msg = 'Lorem ipsum dolor sit amet, consetetur sadipscing elitr, sed diam nonumy\r\n eirmod.';
MessageToast.show(msg);
}
\end{lstlisting}

Das letzte wichtige OpenUI5-Fragment stellt die Datei i18n.properties dar. Sie ist für Webanwendungen im internationalen Umfeld sehr wichtig. Der Entwickler kann mehrere i18n-Dateien mit Übersetzungen in diversen Sprachen erstellen. Das OpenUI5-Framework stellt auf Basis der Browsersprache sicher, dass die passende Sprache als Anzeigesprache verwendet wird. Die Syntax der Dateibenennung hierbei lautet i18n\_XX.properties, wobei das XX für das Länderkürzel der Sprache steht (z.B. Deutsch: i18n\_DE.properties).\autocites[vgl.][126\psqq]{Antolovic2014}