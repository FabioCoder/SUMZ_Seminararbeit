%!TEX root = ../dokumentation.tex

%https://www.online-textbuero.de/953/wie-schreibe-ich-ein-fazit/

\chapter{Fazit}
\vskip -3em
\textbf{Autor: Sebastian Greulich}

Im Zuge dieser Seminararbeit sollten die Webfrontendtechnologien Angular, ReactJS sowie OpenUI5 mit dem Ziel analysiert und verglichen werden um zu überprüfen, ob Angular das passende Framework für die Anwendung „business horizon“ darstellt. 

In der Analyse und dem Vergleich konnten verschiedene Stärken und Schwächen der einzelnen Technologien erarbeitet werden. Es wurde festgestellt, dass sich Angular insgesamt am besten für die Anwendung eignet. Jedoch hat das Framework React Geschwindigkeitsvorteile bei schnellen Datenänderungen auf der Oberfläche. Deshalb lautet die Empfehlung, dass der vorhandene Code belassen wird. Eine Ausnahme hierbei bildet der Einzelfall, dass die Geschwindigkeitsvorteile von React den resultierenden Mehraufwand durch das Umschreiben des Codes rechtfertigen. React wird ebenfalls empfohlen, wenn bei einer Erweiterung von „business horizon“ häufige Anzeigedatenänderungen auftreten. Im Standardfall sollte Angular verwendet werden.

Somit wurde das Ziel dieser Seminararbeit erfüllt und mit der Weiterverwendung von Angular eine fundierte Handlungsempfehlung ausgesprochen.