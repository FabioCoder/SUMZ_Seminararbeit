%!TEX root = ../dokumentation.tex

%https://www.online-textbuero.de/953/wie-schreibe-ich-ein-fazit/

\chapter{Projektbezogener Vergleich der Webframeworks}
\vskip -3em
\textbf{Autor: Sebastian Greulich}

\section{Vergleichende Betrachtung}\label{sec:vergl}

Nachdem in den vorherigen drei Kapiteln die einzelnen Technologien detailliert beschrieben wurden, werden sie nun miteinander verglichen.

Das erste Vergleichskriterium bildet der Umfang an Funktionen welche direkt im Framework integriert sind. Alle drei Webframeworks erlauben es, Daten komfortabel auf der Weboberfläche anzuzeigen. Jedoch bieten nur Angular und OpenUI5 die Möglichkeit im Standard eine Datenquelle mittels „Two-Way-Binding“ anzubinden(siehe \autoref{sec:aTemp} und \autoref{sec:oDateien}). Bei React muss hierfür eine zusätzliche Komponente nachinstalliert werden(siehe \autoref{sec:rWurz}). Ähnlich verhält es sich bei dem Routing zwischen mehreren Seiten der Anwendung. Angular und OpenUI5 bieten Möglichkeiten hierzu in ihrem Standard, bei React muss hierfür wiederum eine zusätzliche Router-Implementierung eingebunden werden(siehe \autoref{sec:rEntw} und \autocites[vgl.][8]{Zeigermann.2016}). 

Es gibt darüber hinaus weitere Unterschiede bei der empfohlenen Webarchitektur. Bei Angular und OpenUI5 wird man aufgrund des Frameworkaufbaus zur Verwendung des MVC-Modells gedrängt. Dies wird bei OpenUI5 deutlich, indem schon beim Einrichtungsassistenten die typische MVC-Struktur angelegt wird(siehe \autoref{sec:oDateien}). Auch das Grundkonzept von Angular ist auf MVC ausgerichtet(siehe \autoref{sec:aEinf}). Anders verhält es sich allerdings bei React. Hier wird die Verwendung von Flux empfohlen, bei dem typischerweise die Daten nur in eine Richtung fließen(siehe \autoref{sec:rEntw}). Dies generiert eine schnelle Performance bei React.

Ein weiteres wichtiges Merkmal bei Webframeworks ist die Datenanzeige. Bei OpenUI5 ist die Anbindung an einen Webserver über OData sehr einfach gehalten. Mit nur ein paar wenigen Codezeilen ist es möglich, Daten anzuzeigen und diese auch bei Bedarf zu ändern(siehe \autoref{lst:UI5View}). Jedoch ist man bei den Anzeigemöglichkeiten, welche einfach einzubinden sind, auf den von SAP entwickelten Standard beschränkt. Will man Daten anderweitig anzeigen muss man mit großem Aufwand andere Frameworks einbinden und den Datenaustausch mit diesen manuell implementieren. Bei Angular wird die Datenanzeige über Komponenten gesteuert, welche die Templates mit Inhalt füllen(siehe \autoref{sec:aTemp}). Bei Angular sind oberflächenspezifische Anpassungen komfortabler und schneller möglich als bei OpenUI5, da diese direkt im CSS vorgenommen werden können. React hat bei der Datenanzeige klare Geschwindigkeitsvorteile im Vergleich zu OpenUI5 und Angular, da das Framework die Anzahl der DOM-Manipulationen auf ein Minimum reduziert(siehe \autoref{Rendern}). Daraus folgt, dass der Webbrowser weniger Aktionen ausführen muss. Im Vergleich zu Angular und React wird bei OpenUI5 das Design der Steuerelement vorgegeben und kann nur mit erheblichem Aufwand auf eigne Bedürfnisse angepasst werden(siehe \autoref{sec:oDateien}).   

Bei der Wahl der Entwicklungsumgebung ist man bei Angular und React weitestgehend frei(siehe \autoref{sec:aEntw} und \ref{sec:rEntw}). Man sollte jedoch bei beiden darauf achten, dass die Entwicklung in TypeScript unterstützt wird, falls man diese Programmiersprache verwendet. Bei OpenUI5 ist man in dieser Hinsicht eingeschränkt. Prinzipiell ist jeder Texteditor dafür geeignet, in Eclipse werden jedoch spezielle AddOns angeboten, welche das Erstellen und Entwickeln an einer Anwendung erleichtern(siehe \autoref{sec:oEntw}).

Es gibt weiterhin auch Unterschiede bei den Wahlmöglichkeiten der Programmiersprache. Wie im \autoref{sec:ts} erläutert bietet die von JavaScript abgeleitete Programmiersprache TypeScript zahlreiche Vorteile gegenüber ihrem Ursprung. Die Verwendung von TypeScript ist sowohl in Angular, als auch in React möglich(siehe \autoref{sec:aEinf} und \ref{sec:rEinf}). Somit können mögliche Syntaxfehler schon beim Kompilieren erkannt und die Kompatibilität mit veralteten Webbrowsern sichergestellt werden. Bei OpenUI5 ist TypeScript zur Entwicklung nicht vorgesehen, deshalb wird hierbei das Entwickeln nur in der Webgrundsprache JavaScript empfohlen(siehe \autoref{sec:oEinf}). Diese wird auch von Angular sowie React unterstützt.

Ein weiteres Unterscheidungsmerkmal ist die Lizenzierung. Offiziell stehen alle drei zu vergleichenden Webframeworks unter Open-Source-Lizenzen. Jedoch werden lediglich die Frameworks Angular und React von einer breiten Entwicklergemeinde vorangetrieben. Bei OpenUI5 entwickelt hauptsächlich die SAP SE an Verbesserungen des Frameworks. Dies liegt daran, dass UI5 hauptsächlich in Unternehmen eingesetzt wird und diese ihre Entwicklungen nur selten veröffentlichen. 

Der letzte Unterschied zwischen den Webframeworks ist, dass sowohl Angular, als auch React native Versionen ihres Frameworks anbieten. Somit ist es möglich, mit ihnen native Apps für iOS, Android und WindowsPhone zu entwickeln. Bei OpenUI5 ist diese Möglichkeit nicht vorgesehen.


\section{Handlungsempfehlung}\label{sec:empf}

In diesem Abschnitt werden nun nach dem Vergleich aus dem vorherigen \autoref{sec:vergl} konkrete Empfehlungen zur Umsetzung in „business horizon“ ausgesprochen.

Um eine fundierte Empfehlung für eine Webtechnologie geben zu können muss zunächst die Anwendung auf ihre frontendrelevanten Eigenschaften untersucht werden.



„Business horizon“ umfasst viele unterschiedliche Komponenten und hat zudem noch einen Anmeldebildschirm. Daraus folgt eine Vielzahl von Views zwischen denen gewechselt werden muss.
Nun ist abzuwägen, inwiefern sich die drei betrachteten Webframeworks für dieses Kriterium eignen. Beim Routing haben sowohl Angular, als auch OpenUI5 ihre Stärken. In React muss hierfür eine zusätzliche Komponente nachinstalliert werden.

In „business horizon“ gibt es ausschließlich unidirektionale Webserveranbindungen. OpenUI5 hätte bei bidirektionalen Bindungen seine Vorzüge. Da diese allerdings nur selten benötigt werden sind alle betrachteten Webframeworks in diesem Punkt gleichwertig.

Die Anwendung umfasst zusätzlich viele nicht standardisiert aufgebaute Dialoge. Deshalb sollte das zu verwendende Framework viele Anpassungsmöglichkeiten der Anzeige bieten. Diese Möglichkeiten bieten Angular und React, bei OpenUI5 sind diese mit einem erheblichen Mehraufwand verbunden. Die von der SAP standardisierten Steuermöglichkeiten sind darüber hinaus nicht so eingängig als selbstentwickelte bei Angular und React. Zum Beispiel gibt SAP im Standard lediglich neun Buttontypen vor\autocites[vgl.][]{UI5Doku1}.

Entscheidend ist auch die Hilfestellung bei Problemen. Einen Vorteil für die Frameworks Angular und React stellt die breite Entwicklergemeinde dar. Es existieren zahlreiche Foren, in welchen über aufkommende Probleme beraten wird. UI5 wird zum Großteil von Unternehmen verwendet. Falls diese auf etwaige Probleme stoßen, wenden sie sich an den Support der SAP und teilen die Informationen nicht öffentlich im Internet. Somit würde man mit Angular und React während der Entwicklung einfacher zu Hilfen gelangen.

Falls im weiteren Projektverlauf eine App benötigt wird kann man auf die Frameworks Angular und React zurückgreifen.
Nur diese beiden stellen vorhandene native Versionen bereit.


Einstiegshürden für die Projektmitglieder sollten so gering wie möglich gehalten werden. Bei OpenUI5 existieren Komponenten, welche sich von der typischen Webentwicklung unterscheiden und somit Einarbeitungszeit erfordern. Als Beispiel hierfür kann man die View heranziehen, welche in XML verfasst wird. 
Auch bei React ist eine Einstiegshürde für das Projektteam vorhanden. Da das Architekturmodell Flux verwendet wird, wird auch hier Einarbeitszeit verlangt. Bisher wurde in den Vorlesungen nämlich ausschließlich das Architekturmodell MVC gelehrt.

Ausschlaggebendes Argument stellt das Faktum dar, dass bisher in der Anwendung „business horizon“ vollständig Angular verwendet wurde. Eine Änderung des Frameworks würde einen immensen Reimplementierungsaufwand bedeuten. Bestärkt wird dieses Argument zusätzlich durch die Tatsache, dass bei Angular bisher keinerlei Probleme aufgetreten sind.

Aufgrund der zuvor erörterten Argumente eignet sich Angular im Gesamten als Webfrontend für „business horizon“ am Besten. Es wird zudem empfohlen, dass der bestehende Code belassen wird, außer er bedarf einer schnellen Datenänderung, welche den Mehraufwand zur Implementierung von React rechtfertigt. Jedoch ist die Hürde der Verwendung von React bei Erweiterungen der Anwendung nicht allzu hoch und kann im Einzelfall in Betracht gezogen werden. OpenUI5 kommt für „business horizon“ nicht in Frage, da die zuvor erläuterten Vorteile nicht die Einstiegshürde für die Projektmitglieder rechtfertigen.



