%!TEX root = ../dokumentation.tex

\chapter{Angular}
\vskip -3em
\textbf{Autor: Fabio Krämer}
\label{ch:angular}

\section{Allgemein}

\subsection{Einführung in das Framework}\label{sec:aEinf}

Angular ist ein von Google verwaltetes OpenSource White-Box Framework, mit dem unter anderem clientseitige Single-Page-Applikationen entwickelt werden können. Die erste Version des Angular-Frameworks hat den Namen AngularJS. Alle nachfolgenden Versionen tragen den Namen Angular, da es zwischen der ersten und zweiten Version des Frameworks grundlegende Änderungen gab. 

Das Framework wird kontinuierlich weiterentwickelt. Monatlich soll eine Minor-Version und alle sechs Monate eine Major-Version erscheinen. Die aktuellste stabilste Version von Angular hat die Versionsnummer 7.1.4. \autocites[vgl.][vii\psqq]{Woiwode.2018}[vgl.][3\psqq]{Freeman.2018}

Angular ist in der Sprache TypeScript geschrieben. Angular kann mit TypeScript, JavaScript oder Dart genutzt werden. \autocites[vgl.][vii\psq]{Woiwode.2018}[vgl.][13]{Steyer.2017}

Das in \autoref{MVC} beschriebene MVC Entwurfsmuster kann in einer Angular-Anwendung umgesetzt werden. Die \autoref{fig:AngularMVC} zeigt die einzelnen Bestandteile der Anwendung. Dabei implementiert das Template die View, die Komponente den Controller und das Modell das Model. \autocite[vgl.][34\psqq]{Freeman.2018} Im weiteren Verlauf werden die Bestandteile näher beschrieben. 

\begin{figure}[h]
	\centering
	\includegraphics[width=0.8\linewidth]{Angular-MVC.jpg}
	\caption{Umsetzung des MVC-Musters in Angular} 
	\quelle{ \textcite[][35]{Freeman.2018}}
	\label{fig:AngularMVC}
\end{figure}


\subsection{Vorbereitung der Entwicklungsumgebung}\label{sec:aEntw}
Für die Entwicklung einer Angular-Anwendung wird unter anderem NodeJS, AngularCLI und eine geeignete Entwicklungsumgebung benötigt.  

\label{NodeJS}
Die JavaScript Laufzeitumgebung \textit{NodeJS} ermöglicht unter anderem das Ausführen von JavaScript Code auf dem Server. Einige Tools, die bei der Entwicklung einer Webanwendung eingesetzt werden, verwenden NodeJS. Außerdem bietet NodeJS Pakete mit wiederverwendbarem Code, die über den in integrierten Paketmanager \textit{npm} installiert werden können. Die aktuellste Version kann von \url{https://nodejs.org/} heruntergeladen und installiert werden.  \autocites[vgl.][1\psqq]{Steyer.2017}[vgl.][7\psqq]{Freeman.2018}[vgl.][6\psqq]{Woiwode.2018} 

Das Angular Commandline Interface (kurz: AngularCLI) unterstützt den Entwickler beim Erzeugen und Verwalten einer Angular-Anwendung. AngularCLI erzeugt unter anderem  das Grundgerüst einer Angular-Anwendung und richtet den TypeScript Compiler, Werkzeuge zur Testautomatisierung und den Build-Prozess ein. Die aktuelle Version von AngularCLI kann über den Paketmanager \textit{npm} installiert werden. \autocites[vgl.][1\psqq]{Steyer.2017}[vgl.][7\psqq]{Freeman.2018}[vgl.][6\psqq]{Woiwode.2018} 

%Teilweise ist das von AngularCLI generierte Projekt für den Anwendungsfall nicht ausreichend. Für diesen Fall gibt es weitere Möglichkeiten. \textcite[vgl.][8\psq]{Steyer.2017} beschreibt diese Möglichkeiten in seinem Buch. 

Dies ist die Struktur einer durch AngularCLI erzeugten Angular-Anwendung:

\begin{tabbing}
	mm \= mm \= mmmmmmmmmmmmmmmm \= \kill
	$\vdash$ \textbf{example/} \\ 
	| \> $\vdash$ \textbf{src/}\\ 
	| \> \> $\vdash$  \textbf{app/}\\
	| \> \>  --app.component.css  $\Rightarrow$ \textit{CSS-Datei von AppComponent}\\ 
	| \> \>  --app.component.html  $\Rightarrow$ \textit{Template von AppComponent}\\
	| \> \>  --app.component.ts	 $\Rightarrow$ \textit{Komponente AppComponent}\\
	| \> \>  --app.module.ts  $\Rightarrow$ \textit{Root-Modul AppModule}\\
	%| \> \>  --hello.component.ts  $\Rightarrow$ \textit{Komponente HelloComponent}\\
	| \> \> $\vdash$ \textbf{assets/} \\
	| \> \> $\vdash$ \textbf{environments/} \\
	| \> --index.html\\
	| \> --main.ts\\
	| \> --styles.css \\
	| \> --... \\
	| \> $\vdash$ \textbf{node\_modules/}\\ 
	| \> $\vdash$ \textbf{e2e/}\\   
	| --...\\
\end{tabbing}

Sowohl \textcite[vgl.][3\psqq]{Woiwode.2018} als auch \textcite[vgl.][3\psqq]{Steyer.2017} empfehlen die Verwendung der frei verfügbaren Entwicklungsumgebung Visual Studio Code. Die Entwicklungsumgebung unterstützt die Entwicklung in TypeScript und lässt sich leicht durch Plug-Ins, die die Entwicklung erleichtern sollen, erweitern. Visual Studio Code ist für Linux, Mac und Windows verfügbar und kann unter \url{https://code.visualstudio.com/Download}  heruntergeladen werden.


\section{Grundkonzepte}
In diesem Kapitel werden die Grundkonzepte von Angular näher beschrieben. Darüber hinaus bietet Angular weitere Konzepte wie den Router, der Ansichten abhängig vom Zustand einer Anwendung lädt. Angular kann zudem auf weiteren Plattformen eingesetzt werden. Mit NativeScript lassen sich beispielsweise Native Apps für Android, IOS und Windows Phone entwickeln.\autocites[vgl.][109]{Steyer.2017}[vgl.][431-440]{Woiwode.2018} Diese weiteren Konzepte werden unter anderem in \textcites{Woiwode.2018}{Steyer.2017} näher erläutert.
\label{NativeScript}


\label{BeschreibungHelloBsp}
Zur Beschreibung der Grundkonzepte wird die Beispielanwendung in \autoref{fig:AngularAnwendung} verwendet. Im Input-Feld der Anwendung kann ein Name eingegeben werden. Dieser Name wird zur Begrüßung in der Überschrift verwendet. Eine Änderung des Namens im Input-Feld bewirkt eine Veränderung des Namens in der Überschrift.

\begin{figure}[h]
	\centering
	\includegraphics{angular-anwendung.png}
	\caption{Screenshot Angular-Beispielanwendung} 
	\label{fig:AngularAnwendung}
\end{figure}


\subsection{Angular-Module}

%%Funktion von Modulen
Ein Angular-Modul fasst zusammengehörige Codeeinheiten zusammen und ermöglicht die Strukturierung einer Anwendung. Eine Angular-Anwendung kann dabei aus mehreren Angular-Modulen bestehen. Angular-Module sind nicht mit den in \autoref{sec:der-sprachstandard-ecmascript} vorgestellten JavaScript-Modulen zu verwechseln. \autocites[vgl.][103\psqq]{Steyer.2017}[vgl.][301\psqq]{Woiwode.2018} Die Module einer Angular-Anwendung können in Root-Module, Feature-Module und Shared-Module unterteilt werden.

 
%%Bootstrapping
Jede Angular-Anwendung hat genau ein Root-Modul. Dieses Modul konfiguriert die Angular-Anwendung. Wenn ein Client eine auf Angular basierte Seite anfordert, dann schickt der Server den Inhalt der \textit{index.html} als Antwort an den Client zurück. Der Client führt daraufhin die im HTML-Dokument enthaltenen Skript-Elemente aus. Dabei wird die Angular-Plattform initialisiert und das Root-Modul übergeben. \autocites[vgl.][60]{Steyer.2017}[vgl.][226\psqq]{Freeman.2018}  Das Root-Modul der Beispielanwendung ist das Modul \textit{AppModule} siehe \autoref{lst:AppModule}. 

%%Feature-Module und Shared-Module
Feature Module ermöglichen die Gruppierung einer Anwendung in Anwendungsfällen. Mithilfe von Shared-Module können die Teile einer Anwendung zusammengefasst werden, die unabhängig vom Anwendungsfall verwendet werden können. \autocites[vgl.][528\psqq]{Freeman.2018}[vgl.][]{Google.c}[vgl.][105\psqq]{Steyer.2017}

%Dekorator und Eigenschaften
Module werden im Allgemeinen durch den Dekorator \textit{@NgModule} gekennzeichnet. Ein Modul kann verschiedene weitere Module über die Eigenschaft \textit{import} importieren und damit die bereitgestellten Funktionalitäten verwenden. Die vom Modul verwendeten Direktiven, Komponenten und Pipes werden in der Eigenschaft \textit{declarations} angegeben. Jedes Root-Modul muss die Eigenschaft \textit{bootstrap} führen. Diese Eigenschaft spezifiziert die Komponente, die beim Starten der Anwendung geladen werden soll. 

%% Beschreibung am Beispiel
Das Modul \textit{AppModule} in \autoref{lst:AppModule} importiert die Module \textit{NgModule}, \textit{BrowserModule} und \textit{FormsModule} und deklariert die zugehörigen Komponenten \textit{AppComponent} und \textit{HelloComponent}. Beim Starten der Anwendung soll die Komponente \textit{AppComponent} aufgerufen werden.

\begin{lstlisting}[caption=Das Root-Module in der Datei app.module.ts, label=lst:AppModule, language=Java]
import { NgModule } from '@angular/core';
import { BrowserModule } from '@angular/platform-browser';
import { FormsModule } from '@angular/forms';

import { AppComponent } from './app.component';
import { HelloComponent } from './hello.component';

@NgModule({
	imports:      [ BrowserModule, FormsModule ],
	declarations: [ AppComponent, HelloComponent ],
	bootstrap:    [ AppComponent]
})
export class AppModule { }

\end{lstlisting}

\subsection{Angular-Komponenten}

%%Beschreibung von Komponenten
Komponenten sind Klassen, die Daten und Logik zur Anzeige in den zugehörigen Templates bereitstellen. Diese ermöglichen die Aufteilung einer Angular-Anwendung in logisch getrennte Teile. \autocite[vgl.][401]{Freeman.2018} 

%%Dekorator und Eigenschaften von Komponente
Eine Komponente wird durch den Dekorator \textit{@Component} gekennzeichnet und kann über verschiedene Dekorator-Eigenschaften konfiguriert werden. Die Eigenschaft \textit{selector} identifiziert das HTML-Element, dass durch diese Komponente repräsentiert wird. Zur Anzeige der bereitgestellten Daten kann entweder ein Inline-Template \textit{template} definiert oder auf ein externes Template \textit{templateUrl} verwiesen werden. \autocites[vgl.][]{Google.b}[vgl.][405]{Freeman.2018}[vgl.][47\psqq]{Steyer.2017}

%%Beschreibung des Beispiels
Die Beispielanwendung enthält die Komponenten \textit{AppComponent} (siehe \autoref{lst:AppComponentTs}) und \textit{HelloComponent} (siehe \autoref{lst:HelloComponentTs}).

\begin{lstlisting}[caption=Die Komponente AppComponent in der Datei app.component.ts, label=lst:AppComponentTs, language=Java]
import { Component } from '@angular/core';

@Component({
selector: 'my-app',
	templateUrl: './app.component.html',
	styleUrls: [ './app.component.css' ]
})
export class AppComponent  {
	name: string;
}
\end{lstlisting}

\begin{lstlisting}[caption=Die Komponente HelloComponent in der Datei hello.component.ts, label=lst:HelloComponentTs, language=Java]
import { Component, Input } from '@angular/core';

@Component({
	selector: 'hello',
	template: `<h1>Hello {{name}}!</h1>`,
	styles: [`h1 { font-family: Lato; }`]
})
export class HelloComponent  {
	@Input() name: string;
}
\end{lstlisting}

\subsection{Templates}\label{sec:aTemp}
%%Beschreibung von Templates
Zur Darstellung von Komponenten nutzt Angular Templates. Ein Template besteht aus HTML Code, der um Angular spezifische Konzepte wie Direktiven, Datenbindungsausdrücke und Pipes erweitert wird. \autocites[vgl.][]{Google.b}[vgl.][52]{Steyer.2017} 

%% Beschreibung der unterschiedlichen Direktiven
Mit Direktiven kann einem Element zusätzliches Verhalten hinzugefügt werden. \autocites[vgl.][265]{Steyer.2017}[vgl.][401]{Freeman.2018} In Angular werden folgende drei Arten von Direktiven unterschieden. \autocite[vgl.][]{Google.}

\begin{itemize}
	\item Strukturelle Direktiven 
	\item Attribut-Direktiven
	\item Komponenten
\end{itemize}

%%Strukturelle Direktive
Die strukturellen Direktiven ändern die Struktur des zugehörigen HTML-Elements, indem sie HTML-Elemente hinzufügen oder entfernen. Hierfür verwenden die strukturellen Direktiven Templates, die beliebig oft gerendert werden. \autocites[vgl.][269\psqq]{Steyer.2017}[vgl.][365]{Freeman.2018} Beispiele für strukturellen Direktiven aus Angular sind \autocite[vgl.][261\psqq]{Freeman.2018}:
\begin{description}
	\item [ngIf] Fügt dem HTML-Dokument Inhalt hinzu, wenn die Bedingung wahr ist. 
	\item [ngfor] Fügt für jedes Item einer Datenquelle den gleichen Inhalt dem HTML-Dokument hinzu.
	\item [ngSwitch] Fügt dem HTML-Dokument, abhängig vom Wert eines Ausdrucks, Inhalt hinzu.
\end{description} 

%%Attribut-Direktive
Das Verhalten und das Aussehen des zugehörigen HTML-Elements kann durch die Attribut-Direktiven verändert werden. Diese Direktiven fügen oder entfernen dem zugehörigen HTML-Element Attribute. \autocite[vgl.][339]{Freeman.2018} Beispiele für Attribut-Direktiven aus Angular-JS \autocite[vgl.][249\psqq]{Freeman.2018}:
\begin{description}
	\item [ngStyle] Mit dieser Direktive können unterschiedliche Style-Eigenschaften dem Element hinzugefügt werden.
	\item [ngClass] Weißt dem Element ein oder mehrere Klassen hinzu. 
\end{description}

%%Komponente Direktive
Mittels einer Komponente kann einem HTML-Element eine View hinzugefügt werden. Komponenten sind nämlich Direktiven mit einer eigenen View. \autocites[vgl.][265]{Steyer.2017}

Die von Angular bereitgestellten Direktiven (engl. Built-In Directives) können durch selbst entwickelte Direktiven erweitert werden. \autocite[vgl.][261]{Freeman.2018}

%%Datenbindungsausdrücke
Der Datenaustausch zwischen der Komponente und dem Template erfolgt durch Datenbindungsausdrücke. Ein Datenbindungsausdruck bindet einen JavaScript-Ausdruck an ein Ziel. Das Ziel kann entweder eine Attribut-Direktive oder eine Eigenschaft des zugehörigen HTML-Elements sein. Der JavaScript-Ausdruck ermöglicht den Zugriff auf die Eigenschaften und Methoden der Komponente. \autocites[vgl.][237\psqq]{Freeman.2018}[vgl.][52\psq]{Steyer.2017} [vgl.][]{Google.d} Es gibt insgesamt drei Arten von Data-Bindings, die anhand der Flussrichtung der Daten unterschieden werden können. 

\begin{table}
\begin{tabular}{|>{\raggedright\arraybackslash}p{3cm}|>{\raggedright\arraybackslash}p{3.2cm}|>{\raggedright\arraybackslash}p{6.5cm}|}
	\hline
	\textbf{Richtung}&\textbf{Syntax}&\textbf{Verwendung}\\
	\hline 
	One-way Komponente -> View &\{\{Ausdruck\}\} \lbrack Ziel\rbrack =\dq Ausdruck\dq  & Interpolation, Eigenschaft, Attribut, Klasse, Style\\ 
	\hline 
	One-way Komponente <- View &(Ziel)=\dq Ausdruck\dq&Events\\ 
	\hline 
	Two-way&\lbrack (Ziel)\rbrack =\dq Ausdruck\dq&Formular\\ 
	\hline 
\end{tabular}
\caption{Arten von Data-Bindings}
\label{tab:DataBinding}
\end{table}

%%Pipes
Mit Pipes können die  Daten vor der Ausgabe sortiert, formatiert oder gefiltert werden. \autocite[vgl.][83\psqq]{Steyer.2017}

\begin{lstlisting}[caption=Das Template in der Datei app.component.html, label=lst:AppComponentHTML, language=HTML]
<hello name="{{name}}"></hello>
<label for="name">Name:</label>
<input name="name" [(ngModel)]="name">
\end{lstlisting}

\subsection{Services}
Laut \textcite[vgl.][474]{Freeman.2018} kann jedes Objekt, dass durch Dependency Injection verwaltet und verteilt wird, als Service bezeichnet werden. Services stellen wiederverwendbare Routinen oder Daten zur Verfügung, die von Direktiven, Komponenten, weiteren Services und Pipes verwendeten werden können. \autocites[vgl.][467\psqq]{Freeman.2018}[vgl.][89]{Steyer.2017}

Services verringern die Abhängigkeiten zwischen den Klassen, durch Dependency Injection. Hierdurch können Beispiel Unit-Tests einfacher durchgeführt werden. \autocite[vgl.][469]{Freeman.2018} 

%%Wie kann ein Service verwendet werden? 
%%Registrierung
Um einen Service zu verwenden, muss dieser entweder global in einem Modul oder lokal in einer Komponente registriert werden. Dies geschieht durch Einrichten eines Providers beim Modul oder der Komponente. Der Provider verknüpft ein Token mit einem Service. Ein Service kann eine Klasse, ein Wert, eine Funktion, eine Factory oder eine Weiterleitung sein. Aus diesem Grund gibt es unterschiedliche Provider.

%%Ergebnis der Registrierung
Sobald ein Service global registrierte wurde, steht dieser auch in weiteren Modulen zur Verfügung. Ein lokal registrierter Service kann dahingegen nur von der jeweiligen Komponente und den direkten und indirekten Kindkomponenten verwendet werden. 

%%Nutzung
Zur Nutzung eines Services muss die jeweilige Klasse den Service importieren und im Konstruktor deklarieren. Beim Erzeugen einer Instanz der Klasse injiziert Angular den jeweiligen Service.\autocites[vgl.][92\psqq]{Steyer.2017}[vgl.][474\psqq]{Freeman.2018}